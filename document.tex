% Vorlage für eine Bachelorarbeit
% Siehe auch LaTeX-Kurs von Mathematik-Online
% www.mathematik-online.org/kurse
% Anpassungen für die Fakultät für Mathematik
% am KIT durch Klaus Spitzmüller und Roland Schnaubelt im Dezember 2011

\documentclass[11pt,a4paper,titlepage]{scrartcl}
% scrartcl ist eine abgeleitete Artikel-Klasse im Koma-Skript
% zur Kontrolle des Umbruchs Klassenoption draft verwenden

\usepackage{array}
% die folgenden Packete erlauben den Gebrauch von Umlauten und ß
% in der Latex Datei
\usepackage[utf8]{inputenc}
% \usepackage[latin1]{inputenc} %  Alternativ unter Windows
\usepackage[T1]{fontenc}
\usepackage[ngerman]{babel}
\usepackage[babel,german=quotes]{csquotes}
\usepackage[pdftex]{graphicx}
\usepackage{latexsym}
\usepackage{amsmath,amssymb,amsthm}
\usepackage[style=alphabetic,backend=biber]{biblatex}
\usepackage{pdflscape}

\bibliography{bib/Quellen}


% Abstand obere Blattkante zur Kopfzeile ist 2.54cm - 15mm
\setlength{\topmargin}{-15mm}
\usepackage[onehalfspacing]{setspace}

% Umgebungen für Definitionen, Sätze, usw.
% Es werden Sätze, Definitionen etc innerhalb einer Section mit
% 1.1, 1.2 etc durchnummeriert, ebenso die Gleichungen mit (1.1), (1.2) ..

\newtheorem{Satz}{Satz}[section]
\newtheorem{Definition}[Satz]{Definition}     
\newtheorem{Lemma}[Satz]{Lemma}	
                  
\numberwithin{equation}{section} 

% einige Abkuerzungen
\newcommand{\C}{\mathbb{C}} % komplexe
\newcommand{\K}{\mathbb{K}} % komplexe
\newcommand{\R}{\mathbb{R}} % reelle
\newcommand{\Q}{\mathbb{Q}} % rationale
\newcommand{\Z}{\mathbb{Z}} % ganze
\newcommand{\N}{\mathbb{N}} % natuerliche

\title{Bachelorarbeit}
\author{Andre Weinkötz (14985714)}
\date{15. Februar 2018}


\begin{document}
  % Keine Seitenzahlen im Vorspann
  \pagestyle{empty}

  % Titelblatt der Arbeit
  \begin{titlepage}

\begin{center}
	\includegraphics[scale=0.20]{img/hm-logo.eps}
\end{center}
 \bigskip

 \begin{center} \large 
    
    Bachelorarbeit im Studiengang B.Sc. Wirtschaftsinformatik
    \vspace*{2.5cm}
\end{center}
\begin{center}
	    {\huge Realisierung einer bidirektionalen Chat-Anwendung durch HTML5 WebSockets mit Java und Angular 4 zum Einsatz im seminaristischen Kontext}
	% Echtzeit-Anwendungen im Web durch WebSockets
\end{center}

    
    \vspace*{2.0cm}
 \begin{center}
    Andre Weinkötz \bigskip
    
    

    15. Februar 2018
    \vspace*{2.5cm}
    
    

    Fakultät für Informatik und Mathematik \\
	Hochschule München\bigskip
	
	Betreuer: Prof. Dr. Mandl 
	
	
  \end{center}
\end{titlepage}


  % Inhaltsverzeichnis
 \tableofcontents

\newpage
%\textbf{ABSTRACT}
%\newpage
  % Ab sofort Seitenzahlen in der Kopfzeile anzeigen
  \pagestyle{headings}

\section{Einleitung}\label{sec:Einleitung}
Die Anforderungen an Webanwendungen haben sich in den vergangenen Jahren stark verändert. Mobile Geräte wie Smartphones oder Tablets ersetzen stationäre Systeme, Webanwendungen sollen zur Kollaboration eingesetzt werden und Buchungssysteme oder Finanzanwendungen verlangen die Bearbeitung tausender Anfragen mit minimaler Verzögerung. Das Hypertext Transfer Protocol (HTTP) bietet hier keine zufriedenstellende Lösung. Um den Anforderungen an moderne Webanwendungen gerecht zu werden, spezifizierten das World Wide Web Consortium (W3C) und die Internet Engineering Taskforce (IETF) das WebSocket-Protokoll mit zugehöriger JavaScript-API. \\

\noindent Die Entwickler des WebSocket Protokolls machten es sich zur Aufgabe, einen Mechanismus zu schaffen, der es browserbasierten Anwendungen ermöglicht bidirektional zu kommunizieren ohne dabei auf mehrere HTTP Verbindungen zu öffnen \autocite{fette_websocket_2011}. Dabei sollte auf zusätzliche Methoden - wie beispielsweise den Einsatz von XMLHttpRequests, iframes oder long polling - verzichtet werden.

\newpage
\section{Das WebSocket Protokoll}\label{sec:WebSocketProtokoll}
Die Arbeit an der Spezifikation des WebSocket Protokolls begann bereits 2009, als es von Google in Zusammenarbeit mit Apple, Microsoft und Mozilla im Rahmen ihrer Kooperation WHATWG\footnote{Web Hypertext Application Technology Working Group unter: www.whatwg.org} der Internet Engineering Task Force (IETF) vorgeschlagen wurde. Bis Mai 2010 wurden noch zahlreiche Verbesserungen hinzugefügt, bis es schließlich in Version 76 \autocite{hickson_websocket_2010} im August 2010 an die BiDirectional or Server-Initiated HTTP (HyBi) Task Force der IETF zur Weiterentwicklung übergeben wurde \autocite{fette_websocket_2010}. Neben vielen anderen Fortschritten wurde das Protkoll um die Möglichkeit erweitert binäre Dateien auszutauschen, sowie Sicherheitslücken geschlossen, die in Verbindung mit Proxy-Servern entstehen konnten. Im Dezember 2010 wurde das WebSocket Protokoll von der IETF zu dem Request for Comment (RFC) 6455 erklärt \autocite{fette_websocket_2011}.
\subsection{Aufbau und Funktionsweise}\label{subsec:wsAufbau}
Der für die Spezifikation des WebSocket Protokolls zuständige RFC 6455 

\subsection{Vor- und Nachteile gegenüber Request}
\subsection{Einsatzgebiete von WebSockets}

\section{Die WebSocket API}\label{sec:WebSocketAPI}
\subsection{Client-seitige Implementierung}
\subsection{Server-seitige Implementierung}

\section{Fachliche Anforderungen}
\subsection{Analyse des DaKo-Frameworks}
\subsection{Anforderungen an WebSocket-Chat}
\subsection{Prototypische Implementierung}

\section{Technische Anforderungen}
\subsection{Entwurf des WebSocket-Chats}
\subsection{Vergleich der Implementierungsformen}
\subsection{Implementierung des WebSocket-Chats}

\section{Evaluation}
\subsection{Test-Umgebung}
\subsection{Leistungsanalyse WebSocket-Chat}

\section{Zusammenfassung und Ausblick}
% Unterschrift (handgeschrieben)

\newpage
\pagenumbering{Roman}
\printbibliography

%\newpage
%\listoffigures
%\listoftables

\end{document}
