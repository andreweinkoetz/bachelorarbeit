% Vorlage für eine Bachelorarbeit
% Siehe auch LaTeX-Kurs von Mathematik-Online
% www.mathematik-online.org/kurse
% Anpassungen für die Fakultät für Mathematik
% am KIT durch Klaus Spitzmüller und Roland Schnaubelt im Dezember 2011

\documentclass[11pt,a4paper,titlepage]{scrartcl}
% scrartcl ist eine abgeleitete Artikel-Klasse im Koma-Skript
% zur Kontrolle des Umbruchs Klassenoption draft verwenden

\usepackage{array}
% die folgenden Packete erlauben den Gebrauch von Umlauten und ß
% in der Latex Datei
\usepackage[utf8]{inputenc}
% \usepackage[latin1]{inputenc} %  Alternativ unter Windows
\usepackage[T1]{fontenc}
\usepackage[ngerman]{babel}
\usepackage[babel,german=quotes]{csquotes}
\usepackage[pdftex]{graphicx}
\usepackage{latexsym}
\usepackage{amsmath,amssymb,amsthm}
\usepackage[style=authoryear,backend=biber]{biblatex}
\usepackage{pdflscape}

%\bibliography{bib/test}


% Abstand obere Blattkante zur Kopfzeile ist 2.54cm - 15mm
\setlength{\topmargin}{-15mm}


% Umgebungen für Definitionen, Sätze, usw.
% Es werden Sätze, Definitionen etc innerhalb einer Section mit
% 1.1, 1.2 etc durchnummeriert, ebenso die Gleichungen mit (1.1), (1.2) ..

\newtheorem{Satz}{Satz}[section]
\newtheorem{Definition}[Satz]{Definition}     
\newtheorem{Lemma}[Satz]{Lemma}	
                  
\numberwithin{equation}{section} 

% einige Abkuerzungen
\newcommand{\C}{\mathbb{C}} % komplexe
\newcommand{\K}{\mathbb{K}} % komplexe
\newcommand{\R}{\mathbb{R}} % reelle
\newcommand{\Q}{\mathbb{Q}} % rationale
\newcommand{\Z}{\mathbb{Z}} % ganze
\newcommand{\N}{\mathbb{N}} % natuerliche

\title{Bachelorarbeit}
\author{Andre Weinkötz (14985714)}
\date{15. Februar 2018}


\begin{document}
  % Keine Seitenzahlen im Vorspann
  \pagestyle{empty}

  % Titelblatt der Arbeit
  \begin{titlepage}

\begin{center}
	\includegraphics[scale=0.20]{img/hm-logo.eps}
\end{center}
 \bigskip

 \begin{center} \large 
    
    Bachelorarbeit im Studiengang B.Sc. Wirtschaftsinformatik
    \vspace*{2.5cm}
\end{center}
    {\huge Realisierung einer bidirektionalen Chat-Anwendung durch HTML5 WebSockets mit Java und Angular 4 zum Einsatz im seminaristischen Kontext}
   % Echtzeit-Anwendungen im Web durch WebSockets
    
    \vspace*{2.0cm}
 \begin{center}
    Andre Weinkötz \bigskip
    
    

    15. Februar 2017
    \vspace*{2.5cm}
    
    

    Fakultät für Informatik und Mathematik \\
	Hochschule München\bigskip
	
	Betreuer: Prof. Dr. Mandl 
	
	
  \end{center}
\end{titlepage}


  % Inhaltsverzeichnis
 \tableofcontents

\newpage
%\textbf{ABSTRACT}
%\newpage
  % Ab sofort Seitenzahlen in der Kopfzeile anzeigen
  \pagestyle{headings}

\section{Einleitung}
Die Anforderungen an Webanwendungen haben sich in den vergangenen Jahren stark verändert. Mobile Geräte wie Smartphones oder Tablets ersetzen stationäre Systeme, Webanwendungen sollen zur Kollaboration eingesetzt werden und Buchungssysteme oder Finanzanwendungen verlangen die Bearbeitung tausender Anfragen mit minimaler Verzögerung. Das Hypertext Transfer Protocol (HTTP) bietet hier keine zufriedenstellende Lösung. Um den Anforderungen an moderne Webanwendungen gerecht zu werden, spezifizierten das World Wide Web Consortium (W3C) und die Internet Engineering Taskforce (IETF) das WebSocket-Protokoll mit zugehöriger JavaScript-API. \\

\noindent Das Ziel der geplanten Bachelorarbeit ist der Einsatz dieser Technologie als didaktisches Mittel im Rahmen der Lehrveranstaltung “Web-Techniken” an der Hochschule München. Die Grundlage bildet dabei eine Chat-Anwendung aus der Lehrveranstaltung “Datenkommunikation”, die von den Studierenden im dritten Fachsemester als Studienarbeit bearbeitet wird.\\

\noindent Zu Beginn werden die technischen Aspekte der WebSockets analysiert. Diese umfassen sowohl die konzeptionelle Spezifikation von WebSockets, als auch deren Implementierung in Java. Dabei sollen traditionelle Webtechniken und das neue Konzept gegenübergestellt werden, um die jeweiligen Vor- und Nachteile aufzuzeigen. Zusätzlich wird untersucht, welche Anwendungsfälle das größte Potential bei dem Einsatz von WebSockets bergen. \\

\noindent Der Kern der Arbeit handelt vom Umbau der bestehenden Anwendung. Hier soll eine Analyse der bestehenden Anwendung erfolgen, um daraus die Anforderungen an die modifizierte Version abzuleiten. Die Umsetzung dieser Anforderungen erfolgt zunächst konzeptionell und wird anschließend  unter Berücksichtigung des erstellten Konzepts implementiert. Als serverseitige Programmiersprache wird Java beibehalten. Der Programmteil zur Leistungsmessung wird ebenfalls in Java verfasst. Die Implementierung der clientseitigen Chat-Anwendung erfolgt in JavaScript.\\

\noindent Zum Einsatz der Anwendung im Rahmen einer Lehrveranstaltung wird ein Lehrkonzept erstellt, das als zukünftige Aufgabenstellung oder Anleitung fungiert. Abschließend wird die Leistungsfähigkeit beider Anwendungen geprüft und bewertet. Zudem wird der Einsatz der modifizierten Anwendung als interaktive Methode einer Lehrveranstaltung evaluiert.
\section{Die Technologie WebSockets}
\subsection{Aufbau und Funktionsweise}
\subsection{Vor- und Nachteile gegenüber Request}
\subsection{Einsatzgebiete von WebSockets}

\section{Fachliche Anforderungen}
\subsection{Analyse des DaKo-Frameworks}
\subsection{Anforderungen an WebSocket-Chat}
\subsection{Prototypische Implementierung}

\section{Technische Anforderungen}
\subsection{Entwurf des WebSocket-Chats}
\subsection{Vergleich der Implementierungsformen}
\subsection{Implementierung des WebSocket-Chats}

\section{Evaluation}
\subsection{Leistungsanalyse WebSocket vs. TCPStream}
\subsection{Bewertung des WebSocket-Chats als Lehrmittel}

\section{Zusammenfassung und Ausblick}
% Unterschrift (handgeschrieben)

%\newpage

%\printbibliography

%\newpage
%\listoffigures
%\listoftables

\end{document}
